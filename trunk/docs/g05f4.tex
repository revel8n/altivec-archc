\documentclass[twocolumn]{article}

\usepackage[utf8]{inputenc}
\usepackage{graphicx}
\usepackage[brazil]{babel}
\usepackage{palatino}
\usepackage{fullpage}

\newcommand{\tech}[1]{\textit{#1}}
\newcommand{\who}[1]{\textbf{ (#1)}}

\begin{document}

\title{Projeto de MC723 - Grupo 05 - Relatório da Fase 4\\ Modelagem das instruções AltiVec no ArchC}
\author{
Caio Marcelo de Oliveira Filho, 015599 \and
Helder dos Santos Ribeiro, 033245 \and
Ribamar Santarosa de Sousa, 017209 \and 
Tarcísio Genaro Rodrigues, 017391
}

\date{}
\maketitle

\section{Objetivos planejados para a fase 4}

\begin{itemize}

% Listar os objetivos (ta no relatorio anterior) e ir marcando OK ou nao OK.. Veja o
% modelo de como isso eh feito no relat anterior.

%\item Lalalalla. \textbf{OK.}

\end{itemize}

Comentários sobre o que foi realizado:

\subsection{Instruções implementadas}

%A meta de 91 instruções implementadas e testadas foi atingida. O foco desta vez
%foram as instruções mais complexas com operações sobre inteiros e lógicas. Os
%grupos gerais e seus implementadores foram:

\begin{itemize}

% Aqui voce lista as da ultima fase somente

%\item Vector Integer Multiply-Add/Sum Instructions (9 instruções) \who{Ribamar}
%\item Vector Integer Average Instructions (6 instruções) \who{Ribamar}
%\item Vector Integer Subtract Instructions (10 instruções) \who{Helder}
%\item Vector Permute and Select Instructions (2 instruções) \who{Helder}
%\item Vector Shift Instructions (3 instruções) \who{Helder}
%\item Vector Integer Rotate and Shift Instructions (12 instruções) \who{Tarcísio}
%\item Vector Integer Add Instructions (3 instruções) \who{Tarcísio}
%\item Instruções órfãs de vários grupos (splat, mult, etc.) (~15 instruções) \who{Caio}


% FIXME: falar que conversamos com o professor e retiramos Floating Point desta entrega

\subsection{Testes}

% Falar um ou dois parágrafos sobre os testes (temos muitos! e eles testam bastante coisa)

% OPCIONAL: Se achar valido, coloque um teste (os de vcmp ou algum que tenha
% saturacao sao interessantes)


\section{Conclusão}

% Migueh, tem que ser coisa de 3 paragrafos com conteudo, ideias:

% Comparar o que achava-se que era o trabalho com o que acabou sendo
% (talvez seja a mesma coisa).

% Faz um parágrafo falando dos testes, como eles ajudam na hora que você
% tá implementando. Comenta que seria interessante também a especificação
% do PowerISA ser ilustrada com exemplos de execução (lembra que agente falou
% disso) das instruções e o conteúdo dos registradores envolvidos.

% Nao pudemos aproveitar muito o fato da coisa estar em C++, nao houve tanto
% planejamento para evitar repeticao de codigo quanto gostariamos e/ou quanto
% planejamos os testes e tals.

% Finaliza bonito.

\end{document}

%:wq
