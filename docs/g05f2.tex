\documentclass[twocolumn]{article}

\usepackage[utf8]{inputenc}
\usepackage{graphicx}
\usepackage[brazil]{babel}
\usepackage{palatino}
\usepackage{fullpage}

\newcommand{\tech}[1]{\textit{#1}}
\newcommand{\who}[1]{\textbf{ (#1)}}

\begin{document}

\title{Projeto de MC723 - Grupo 05 - Relatório da Fase 2\\ Modelagem das instruções AltiVec no ArchC}
\author{
Caio Marcelo de Oliveira Filho, 015599 \and
Helder dos Santos Ribeiro, 033245 \and
Ribamar Santarosa de Sousa, 017209 \and 
Tarcísio Genaro Rodrigues, 017391
}

\date{}
\maketitle

\section{Objetivos planejados para a fase 2}

\begin{itemize}

\item Bootstrap de infraestrutura. Mesmo que todos não usem ambientes de
trabalho diferentes do IC (que terá infraestrutura necessária) todos devem se
familiarizar com a infraestrutura do ArchC e como o modelo PowerPC funciona.
\textbf{OK.}

\item Começa e termina a atividade de inclusão das instruções no modelo.
\textbf{OK.}

\item Começa a atividade de desenvolvimento de testes. \textbf{OK.}

\item Começa a atividade de implementação, previsão de 32 instruções, pelo
menos um par Load/Store para facilitar testes. \textbf{OK.}

\item \textbf{Entrega 2}, contendo uma versão com \textbf{32}
instruções funcionando e testes para elas, e um relato do que foi desenvolvido
durante a Fase 2, e um planejamento mais preciso dos grupos de instrução para
Fase 3 e 4. \textbf{OK.}

\end{itemize}

Comentários sobre o que foi realizado:


\subsection{\tech{Patch} para o ArchC suportar \tech{regbank} de 128 bits}

Após conversa com o professor, foi dedicado um tempo para permitir bancos de
registradores. Apesar de apertar um pouco a Fase 2, o resultado foi positivo
pelo conhecimento que foi adquirido do funcionamento interno do ArchC.

A modificação foi feita de forma a permitir futuras extensões, por exemplo:
incluir suporte funcional a registradores de 256 bits seria trivial. Há interesse
do grupo de discutir a solução proposta e executada com o professor para possível
inclusão no código do ArchC.

\subsection{Instruções implementadas}

A meta de 32 instruções implementadas foi atingida. O foco principal foram as
instruções de Load/Store, um subconjunto das instruções de somas vetoriais
(para \tech{bytes}, \tech{halfwords} e \tech{words}), instruções de máximo e
mínimo, e operadores lógicos. Essa amostragem foi abrangente o suficiente para
dar uma noção do escopo das instruções do AltiVec.

Pelo menos 32 instruções tem testes exercitando o funcionamento básico. Os
testes foram escritos em C com \tech{inline assembly} para chamar as instruções
a serem testadas. No momento o foco é no comportamento de cada uma das
instruções (apesar de todos testes dependerem de alguma forma de Load/Store).
Além disso foi feito um teste envolvendo o uso das instrínsicas do GCC para
manipulação de vetores numéricos. Este último abre caminho para futuras
experiências com as opções de auto-vetorização do GCC.

A experiência com as instruções também será útil para na próxima fase ser feito
um esforço de refatorar um pouco o código e tentar minimizar a repetição de
código em diversos lugares. O AltiVec possui \tech{padrões} de instrução
bastante parecidos e um esforço para abstraí-los seria proveitoso.


\section{Próximas fases}

Durante a fase 2 constatou-se que há mais instruções do que tinhamos contado no
planejamento anterior. A contagem que entendemos no momento é de 161
instruções. As metas de instruções serão re-feitas com base neste novo número.

O plano básico para as próximas fases continua o mesmo. Porem duas novas
subatividades apareceram: a primeira é um esforço por parte dos implementadores
de analisar o código das instruções globalmente e tentar refatorar partes
repetidas. Esse esforço terá efeito positivo nas duas próximas fases, então uma
certa folga na meta de instruções será dada a parte de implementação na fase 3
para simplificar a implementação na fase 4.

A segunda subatividade é, no contexto de testes, seria testar se otimizações de
autovetorização incluídas no GCC 4.x geram código que funciona no nosso modelo.
Esse novo teste seria \emph{opcional} e possivelmente ficaria no lugar do teste
de um programa real. Sugestões nessa área são bem-vindas.

A última modificação feita é que vamos estender o prazo da atividade de
desenvolvimento de testes para a Fase 4. A consequência disso é que não mais
teremos todos os testes terminados no fim da Fase 3. O motivo é, apesar de ser
extremamente útil já ter os testes prontos na implementação, o fato do prazo
ser curto e de todos terem que passar pela implementação das instruções torna
mais conveniente que os testes sejam feitos em conjunto com as instruções.

\subsection{Fase 3}

\begin{itemize}

\item Durante essa fase continua o desenvolvimento de testes. \who{todos}

\item Continua a atividade de implementação, implementando mais instruções.
Focar nos grupos de instruções mais complicados. \who{todos, dividindo
em grupos funcionais de instruções}

\item \textbf{Entrega 3 -- 12/Junho}, contendo uma versão com \textbf{91}
instruções funcionando e testes para elas, e um relato do que foi desenvolvido
durante a Fase 3.\who{todos}

\end{itemize}


\subsection{Fase 4}

\begin{itemize}

\item Durante essa fase termina o desenvolvimento de testes. \who{todos}

\item Continua a atividade de implementação, implementando as instruções
restantes. \who{todos, menos para quem estiver fazendo teste opcional se
houver}

\item Começa e termina a atividade de integração final. \who{todos}

\item \emph{(opcional)} Se houver tempo, testar se funcionam as otimizações de
autovetorização do GCC 4.x OU testar a execução de um programa real que use
AltiVec. \who{1 pessoa}

\item \textbf{Entrega 4 -- 26/Junho}, contendo uma versão com todas
\textbf{161} instruções funcionando e testes para elas, e um relato do que foi
desenvolvido durante a Fase 4. \who{todos}

\end{itemize}

\end{document}

%:wq
